%% For double-blind review submission, w/o CCS and ACM Reference (max submission space)
\documentclass[sigplan,screen,10pt]{acmart}
\settopmatter{printfolios=true}
%% For double-blind review submission, w/ CCS and ACM Reference
%\documentclass[sigplan,review,anonymous]{acmart}\settopmatter{printfolios=true}
%% For single-blind review submission, w/o CCS and ACM Reference (max submission space)
%\documentclass[sigplan,review]{acmart}\settopmatter{printfolios=true,printccs=false,printacmref=false}
%% For single-blind review submission, w/ CCS and ACM Reference
%\documentclass[sigplan,review]{acmart}\settopmatter{printfolios=true}
%% For final camera-ready submission, w/ required CCS and ACM Reference
%\documentclass[sigplan]{acmart}\settopmatter{}


%% Copyright information
%% Supplied to authors (based on authors' rights management selection;
%% see authors.acm.org) by publisher for camera-ready submission;
%% use 'none' for review submission.
%%% The following is specific to PEPM '20 and the paper
%%% 'GOOL: A Generic Object-Oriented Language'
%%% by Jacques Carette, Brooks MacLachlan, and Spencer Smith.
%%%
\setcopyright{acmlicensed}
\acmPrice{15.00}
\acmDOI{10.1145}
\acmYear{2026}
\copyrightyear{2026}
\acmISBN{978-...}
\acmConference[ShortTitle]{Proceedings of ...}{Date}{Location}
\acmBooktitle{Proceedings of ...}

%% Bibliography style
\bibliographystyle{ACM-Reference-Format}
%% Citation style
%\citestyle{acmauthoryear}  %% For author/year citations
%\citestyle{acmnumeric}     %% For numeric citations
%\setcitestyle{nosort}      %% With 'acmnumeric', to disable automatic
                            %% sorting of references within a single citation;
                            %% e.g., \cite{Smith99,Carpenter05,Baker12}
                            %% rendered as [14,5,2] rather than [2,5,14].
%\setcitesyle{nocompress}   %% With 'acmnumeric', to disable automatic
                            %% compression of sequential references within a
                            %% single citation;
                            %% e.g., \cite{Baker12,Baker14,Baker16}
                            %% rendered as [2,3,4] rather than [2-4].


%%%%%%%%%%%%%%%%%%%%%%%%%%%%%%%%%%%%%%%%%%%%%%%%%%%%%%%%%%%%%%%%%%%%%%
%% Note: Authors migrating a paper from traditional SIGPLAN
%% proceedings format to PACMPL format must update the
%% '\documentclass' and topmatter commands above; see
%% 'acmart-pacmpl-template.tex'.
%%%%%%%%%%%%%%%%%%%%%%%%%%%%%%%%%%%%%%%%%%%%%%%%%%%%%%%%%%%%%%%%%%%%%%


%% Some recommended packages.

\usepackage[utf8]{inputenc}
\usepackage[T1]{fontenc}
\usepackage{microtype}

\usepackage{booktabs}   %% For formal tables:
                        %% http://ctan.org/pkg/booktabs
\usepackage{subcaption} %% For complex figures with subfigures/subcaptions
                        %% http://ctan.org/pkg/subcaption
\usepackage{listings}
\lstset{columns=flexible}

\usepackage{xargs}                      % Use more than one optional parameter 
%in a new commands
\usepackage{pgfplots}
\pgfplotsset{width=9cm,height=6cm,compat=1.8}

\usepackage{balance}
%
% \usepackage[colorinlistoftodos,prependcaption,textsize=tiny]{todonotes}

% for nice TODO notes
% \newcommandx{\unsure}[2][1=]{\todo[inline,linecolor=red,backgroundcolor=red!25,bordercolor=red,#1]{#2}}
% \newcommandx{\change}[2][1=]{\todo[inline,linecolor=blue,backgroundcolor=blue!25,bordercolor=blue,#1]{#2}}
% \newcommandx{\info}[2][1=]{\todo[inline,linecolor=OliveGreen,backgroundcolor=OliveGreen!25,bordercolor=OliveGreen,#1]{#2}}
% \newcommandx{\improvement}[2][1=]{\todo[inline,linecolor=Plum,backgroundcolor=Plum!25,bordercolor=Plum,#1]{#2}}

%%%%%%%%%%%%%%%%%%%%%
%% Useful abbreviations
\newcommand{\Csharp}{C\#}
\newcommand{\Cplusplus}{C\texttt{++}}

\newcommand{\abbrev}[1]{\textbf{#1}}
\newcommand{\mainstream}{\abbrev{mainstream}}
\newcommand{\readable}{\abbrev{readable}}
\newcommand{\idiomatic}{\abbrev{idiomatic}}
\newcommand{\documented}{\abbrev{documented}}
\newcommand{\oopatterns}{\abbrev{patterns}}
\newcommand{\common}{\abbrev{common}}
\newcommand{\expressivity}{\abbrev{expressivity}}

\pagenumbering{gobble}

\begin{document}

%% Title information
\title{Designing Syntactic Over Languages}         %% [Short %%Title] is optional;
                                        %% when present, will be used in
%                                        %% header instead of Full Title.
%\titlenote{with title note}             %% \titlenote is optional;
%                                        %% can be repeated if necessary;
%                                        %% contents suppressed with 'anonymous'
%\subtitle{(Short Paper)}                     %% \subtitle is optional
%\subtitlenote{with subtitle note}       %% \subtitlenote is optional;
%                                        %% can be repeated if necessary;
%                                        %% contents suppressed with 'anonymous'


%% Author information
%% Contents and number of authors suppressed with 'anonymous'.
%% Each author should be introduced by \author, followed by
%% \authornote (optional), \orcid (optional), \affiliation, and
%% \email.
%% An author may have multiple affiliations and/or emails; repeat the
%% appropriate command.
%% Many elements are not rendered, but should be provided for metadata
%% extraction tools.

%% Author with single affiliation.
%\orcid{nnnn-nnnn-nnnn-nnnn}             %% \orcid is optional
\author{Jacques Carette}
\orcid{0000-0001-8993-9804}
\affiliation{
%  \position{Position1}
  \department{Department of Computing and Software}              %%
  %%\department is recommended
  \institution{McMaster University}            %% \institution is required
  \streetaddress{1280 Main Street West}
  \city{Hamilton}
  \state{Ontario}
  \postcode{L8S 4L8}
  \country{Canada}                    %% \country is recommended
}
\email{carette@mcmaster.ca}          %% \email is recommended

\author{William Farmer}
%\authornote{with author1 note}          %% \authornote is optional;
                                        %% can be repeated if necessary
\affiliation{
  \department{Department of Computing and Software}              %% \department
  \institution{McMaster University}            %% \institution is required
  \streetaddress{1280 Main Street West}
  \city{Hamilton}
  \state{Ontario}
  \postcode{L8S 4L8}
  \country{Canada}                    %% \country is recommended
}
\email{wmfarmer@mcmaster.ca}          %% \email is recommended

\author{Spencer Smith}
%\authornote{with author1 note}          %% \authornote is optional;
%% can be repeated if necessary
\orcid{0000-0002-0760-0987}
\affiliation{
  \department{Department of Computing and Software}              %%
  \institution{McMaster University}            %% \institution is required
  \streetaddress{1280 Main Street West}
  \city{Hamilton}
  \state{Ontario}
  \postcode{L8S 4L8}
  \country{Canada}                    %% \country is recommended
}
\email{smiths@mcmaster.ca}          %% \email is recommended


%% Abstract
%% Note: \begin{abstract}...\end{abstract} environment must come
%% before \maketitle command
\begin{abstract}
  Abstract goes here.
\end{abstract}


%% 2012 ACM Computing Classification System (CSS) concepts
%% Generate at 'http://dl.acm.org/ccs/ccs.cfm'.
\begin{CCSXML}
<ccs2012>
<concept>
<concept_id>10011007.10011006.10011041.10011047</concept_id>
<concept_desc>Software and its engineering~Source code generation</concept_desc>
<concept_significance>500</concept_significance>
</concept>
<concept>
<concept_id>10011007.10010940.10010971.10011682</concept_id>
<concept_desc>Software and its engineering~Abstraction, modeling and modularity</concept_desc>
<concept_significance>300</concept_significance>
</concept>
<concept>
<concept_id>10011007.10011006.10011008.10011009.10011011</concept_id>
<concept_desc>Software and its engineering~Object oriented languages</concept_desc>
<concept_significance>300</concept_significance>
</concept>
</ccs2012>
\end{CCSXML}

\ccsdesc[500]{Software and its engineering~Source code generation}
\ccsdesc[300]{Software and its engineering~Abstraction, modeling and modularity}
%% End of generated code


%% Keywords
%% comma separated list
\keywords{Code Generation, Domain Specific Language, Haskell, Documentation}

\maketitle

\section{Introduction}

\section{Examples}

\subsection{Logic Languages}

\subsection{GOOL}

\subsection{Pandoc}

\subsection{Panbench}

\section{Synthesis}

\section{Principles}

\section{Concluding Remarks}

\balance
\bibliography{References}

%% Appendix
% \appendix
% \section{Appendix}
% 
% Text of appendix \ldots
% 
\end{document}
